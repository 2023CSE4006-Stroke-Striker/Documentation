\section{Requirements}
    \subsection{Front-End}
        \subsubsection{Camera}
            The camera component is a pivotal element of the system, responsible for capturing and recognizing the user's face. It must provide high-quality input images for accurate disease detection, specifically focusing on the early signs of conditions like stroke. The camera should meet the following criteria: 
            \begin{itemize}
                \item High-resolution imaging capability to capture facial details.
                \item Integration with suitable optics and lighting for optimal image quality.
                \item Real-time image capture with minimal latency.
                \item Environmental adaptability to perform effectively in various lighting conditions.
            \end{itemize}
        \subsubsection{Speaker}
            The speaker is a vital component for conveying the results of the face recognition and disease detection process to the users. The speaker should serve the following functions:
            \begin{itemize}
                \item Provide clear and intelligible voice feedback.
                \item Deliver the results of the face photo analysis in a user-friendly manner.
                \item Integrate text-to-speech (TTS) capabilities to communicate results.
                \item Ensure audibility and volume control to suit different user preferences.
            \end{itemize}
    \subsection{Middle-End}
        \subsubsection{Raspberry Pi}
            The Raspberry Pi acts as an intermediary that facilitates the seamless interaction between the front-end of the software, the camera and the speaker, with the back-end of it, the artificial intelligence servers.
            \begin{itemize}
                \item The Raspberry Pi should efficiently relay data between the camera and speaker.
                \item It must be compatible with the chosen camera and speaker hardware.
                \item Adequate computational power to handle image processing, data formatting, and communication with the AI server.
            \end{itemize}
    \subsection{Back-End}
        \subsubsection{Face Recognition AI}
            This is the heart of the system, an AI model designed to recognize facial features and determine whether there are signs of a specific disease, such as stroke. Key considerations for the Face Recognition AI component include:
            \begin{itemize}
                \item Implement a Convolutional Neural Network (CNN) model to effectively process and analyze facial data.
                \item Develop the AI model using Python and PyTorch, taking advantage of their powerful deep learning capabilities.
                \item Train the model on a comprehensive and diverse data set of facial images to ensure robust disease detection.
                \item Aim for a high level of accuracy and sensitivity in disease detection while minimizing false positives.
                \item Ensure the AI can process data in real-time to provide timely results.
                \item Scale the AI system to handle the desired number of users as needed.
            \end{itemize}
