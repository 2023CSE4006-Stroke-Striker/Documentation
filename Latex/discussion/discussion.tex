\section{\textbf{Discussion}}

One notable difficulty encountered was in managing communication within a diverse and geographically dispersed team. As a part of a collaborative project involving team members from different cultural backgrounds, coordinating efforts and maintaining effective communication became a significant challenge. 

To overcome these challenges, the team implemented strategies such as establishing clear communication protocols, utilizing collaborative online platforms for real-time updates, and encouraging team members to provide feedback on the communication processes. Despite the initial difficulties, these measures helped enhance overall team cohesion and productivity.

This teaches the importance of proactive communication and strategic planning to address both technical and non-technical issues in a collaborative work environment.

Also, since we wanted to implement IoT technology using Raspberry Pi, we faced a lot of difficulties due to unfamiliarity with the hardware. In addition to setting up the development environment, we tried to overcome the hardware limitations by using multithreading.

Also, when using various Amazon services for MLops, we had difficulties in preprocessing the data and using the model properly. We tried to overcome this by referring to Amazon's official documentation. As a result, we realized that the dataset and preprocessing for AI are more important than the model.